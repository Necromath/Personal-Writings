\documentclass{article}

\begin{document}

\title{DRP Guidelines 2023}

\maketitle

\section{So, what is DRP?}

DRP stands for Directed Reading Program. This is a program designed for undergraduates who are
interested in learning advanced mathematics in a personalized 1 on 1 setting and are looking to
engage with material that might not be ordinarily offered in the curriculum. Selected students are
paired with graduate students who specialize in one of the students indicated area of interest. The
student and mentor will agree on a project. It can be based on a reading through a book or a paper,
but projects don't strictly have to be these things. 

\section{Expectations for DRP}

\subsection{Student Expectations}

Students are expected to meet with their mentors once a week for an hour. The students should be
spending the necessary time outside of their meetings to focus on DRP related to ensure that their
meetings with their mentors are as productive as possible. At the end of the semester, students are
expected to give a 15-25 minute presentation on their work/project. 


\subsection{Mentor Expectations}

Mentors are expected to meet with their students once a week for an hour and make sure that their
students are making progress on their projects. Mentors should let the organizers know ASAP about
any issue that arises such as a student missing meetings without notice or students not doing work.  

\section{Talks}





\section{So, you want to be accepted to participate in the DRP}

This document summarizes what you can do, as a student, to increase your chances to getting paired
with a graduate student mentor. More often than not, more students apply to the program than there
is enough resources to accommodate. Some of the factors taken into consideration for participating
include the number of available mentors in your area of interest, student seniority, and past DRP
participation. These are out of your control as a student. However, there are many things you could
do as a student to improve your chances.


\subsection{Don't give up!}

Keep applying to the DRP. Continuously applying will naturally improve your chances of selection. If
you didn't get in this semester, you are more likely to be selected for next semester. We will give
higher priority to students who are older, since they have taken more math classes and will have
fewer opportunities to participate. 

\subsection{Take more math classes}

It is easier to pair students who have had more experience and exposure to different kinds of
mathematics. Since, if you know more material, there are more projects and directions you can go
into. It is a lot easier on our mentors if you have taken \bf{at least one proof-based math course} 
This will be important 

\subsection{Tell us why you want to do DRP!}

\end{document}
