\documentclass{article}

\begin{document}
\author{Mauricio Montes}
\title{DRP Guidelines 2023}

\maketitle

\section{So, what is DRP?}

DRP stands for Directed Reading Program. This is a program designed for undergraduates who are
interested in learning advanced mathematics in a personalized 1 on 1 setting and are looking to
engage with material that might not be ordinarily offered in the curriculum. Selected students are
paired with graduate students who specialize in one of the students indicated area of interest. The
student and mentor will agree on a project. It can be based on a reading through a book or a paper,
but projects don't strictly have to be these things. At the end of the semester, all the students give presentations summarizing their work and share the knowledge they obtained during the program.

\section{Expectations for DRP}

\subsection{Student Expectations}

Students are expected to meet with their mentors once a week for an hour. The students should be
spending the necessary time outside of their meetings to focus on DRP related work to ensure that their
meetings with their mentors are as productive as possible. At the end of the semester, students are
expected to give a 15-25 minute presentation on their work/project. 


\subsection{Mentor Expectations}

Mentors are expected to meet with their students once a week for an hour and make sure that their
students are making progress on their projects. Mentors should let the organizers know ASAP about
any issue that arises such as a student missing meetings without notice or students not doing work.
Mentors are also expected to attend the presentations at the end of the semester.

\subsection{Being an effective mentor}

\begin{itemize}
    \item Be patient and understanding. Remember that the student is likely not as experienced as you are.
    \item Be encouraging. It is easy to get discouraged when learning new material. 
    \item Be flexible. The student may not be interested in the project you initially proposed. Be open to changing the project to better suit the student's interests.
    \item Be a good listener. Make sure that you are listening to the student's concerns and questions.
    \item Be a good communicator. Make sure that you are communicating your expectations clearly to the student.
    \item Listening is more important than talking. Students might come with questions that you
      don't know the answer to. That's okay! You can learn together. There is nothing wrong with not
      knowing the answer to a question. This is about guiding the student to find the answer, even
      if it is not immediately apparent to you. 
\end{itemize}


\section{Talks}

At the end of the semester, students are expected to give a short 15-25 minute talk on their work
done with their mentor over the course of the program. Any form of presentation is acceptable,
should be cleared first with the organizers. This talk will be given in front of the other students
in the program and the mentors. This is a great opportunity to practice giving talks in a low
stakes environment. People are encouraged to ask questions during the talk and the will be open to
the public.

\section{How to apply}

Applications will be open at the beginning of the semester. Students will be asked to fill out a
form indicating their mathematical interests. Students will be selected based on their interests and
the availability of mentors. Preference is given to students who have not participated in the DRP
before and students who are closer to graduation, since they have fewer opportunities to
participate. 


\end{document}
