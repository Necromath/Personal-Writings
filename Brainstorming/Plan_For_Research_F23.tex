\documentclass{article}
\usepackage{amsmath}

\begin{document}
\setlength{\parindent}{6pt}


\title{Plan for Research}

\author{Mauricio Montes}

\maketitle

\section{What is it exactly that you are trying to do?}

I am trying to find a way to use the information from the seismic reflection data in a TDA setting.
I want to use the information from the seismic reflection data to create a topological representation of the subsurface.
This representation will be used to find the most likely location of the oil reservoirs. \\ There are
various indicators that can be used to find the location of the oil reservoirs. The most important
of these indicators is the amplitude of the seismic reflection data. The amplitude of the seismic
data is a measure of the strength of the reflection. Directed hydrocarbon indicators (DHI) are a
specific type of amplitude anomaly that are used to find the location of the oil reservoirs. \\ The
idea is that the DHI have some sort of topological signal that can be tracked outside of the noise
from the seismic reflection data. The goal is to find a way to track this topological signal.

\section{Factors for the formation of oil reservoirs}

There are various factors that are needed for the formation of oil reservoirs. The most important of
which are the source rock, the reservoir rock, the seal rock, and the trap. The source rock is the
rock that contains the organic matter that will be transformed into oil. The reservoir rock is the
rock that contains the oil. The seal rock is the rock that prevents the oil from escaping. The trap
is the structure that allows the oil to accumulate. \\ The source rock is usually a shale. The
reservoir rock is usually a sandstone. The seal rock is usually a shale. The trap is usually a fold
or a fault. \\ The source rock is usually a shale because it is a fine grained rock that contains a
lot of organic matter. The reservoir rock is usually a sandstone because it is a coarse grained rock
that has a lot of pore space. The seal rock is usually a shale because it is a fine grained rock
that has a low permeability. The trap is usually a fold or a fault because it is a structure that is
what allows the oil to accumulate. 

\section{Why is it important?}

The oil industry is a very important industry for the world. The oil industry is responsible for a
large part of the world economy. Oil and gas exploration is a very important and expensive part of
the process of recovering oil. The seismic reflection data is the most important tool in the
analysis of the subsurface. There are many factors that are used in determining the location of the
oil reservoir from the seismic detection data. The amplitude of the seismic reflection data is one
of


\section{What is the current state of the art?}

TODO

\section{What is the new idea?}

Apply TDA to the seismic reflection data to find the location of the oil reservoirs. TODO

\section{What is the impact of the new idea?}

Improved accuracy in the location of the oil reservoirs. TODO

\section{Work roadmap}

Need to set up a pipeline for this. There is an implementation that I want to do. At MSRI summer
camp, I encountered a paper with some image analysis that could read what a neuron was thinking in
the last layer of a NN. I want to do something similar with the seismic reflection data. I want to
be able to read what the seismic reflection data is thinking. Ideally, I would like to be able to
have this model be thinking in topological terms. So, I need this model to be robust to noisy data.
Be trained to resist the pulldown that the seismic data comes with. There are already models that
clean the data and adjust for the pulldown/up of the seismic data. My main concern is jury rigging
too many models together. \\ I want to implement CliP Dissect and see if I can get it to work on a
seismic reflection image recognition dataset. This way, I can see if I can get the seismic image model
to think in topological terms. Next, I think that I should try to devise a loss function, or some
other topologically dependent function into a machine learning model that will use the homology
found in the image to find the location of the oil reservoirs.


\bibliographystyle{plain}
\bibliography{Plan_For_Research_F23}

%wiki.aapg.org/Amplitude_(seismic)

%wiki.aapg.org/Direct_hydrocarbon_indicator

%Clip-Dissect: https://arxiv.org/abs/2204.10965



\end{document}
