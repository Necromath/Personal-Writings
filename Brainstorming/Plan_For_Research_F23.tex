\documentclass{article}
\usepackage{amsmath}

\begin{document}
\setlength{\parindent}{6pt}


\title{Plan for Research}

\author{Mauricio Montes}

\maketitle

\section{What is it exactly that you are trying to do?}

I am trying to find a way to use the information from the seismic reflection data in a TDA setting.
I want to use the information from the seismic reflection data to create a topological representation of the subsurface.
This representation will be used to find the most likely location of the oil reservoirs. \\ There are
various indicators that can be used to find the location of the oil reservoirs. The most important
of these indicators is the amplitude of the seismic reflection data. The amplitude of the seismic
data is a measure of the strength of the reflection. Directed hydrocarbon indicators (DHI) are a
specific type of amplitude anomaly that are used to find the location of the oil reservoirs. \\ The
idea is that the DHI have some sort of topological signal that can be tracked outside of the noise
from the seismic reflection data. The goal is to find a way to track this topological signal.
\end{document}
