\documentclass{article}
\begin{document}
\author{Mauricio Montes}
\title{DRP Mentor Guidelines}

\maketitle

\section{What does it mean to be a mentor?}

According to the Cambridge Dictionary, the definition of mentor is "a person who gives a younger or
less experienced person help and advice over a period of time, especially at work or school". Your
job as a mentor is to guide your student through the process of self-learning. 



\section{Mentor Expectations}

Mentors are expected to meet with their students once a week for an hour and make sure that their
students are making progress on their projects. Mentors should let the organizers know ASAP about
any issue that arises such as a student missing meetings without notice or students not doing work.
Mentors are also expected to attend the presentations at the end of the semester.

\subsection{Being an effective mentor}

\begin{itemize}


\item Be patient and understanding. Remember that the student is likely not as experienced as
you are.

\item Be encouraging. It is easy to get discouraged when learning new material. 

\item Be flexible. The student may not be interested in the project you initially
proposed. Be open to changing the project to better suit the student's
interests.

\item Be a good listener. Make sure that you are listening to the student's
concerns and questions.

\item Be a good communicator. Make sure that you are communicating
your expectations clearly to the student.

\item Listening is more important than talking. Students might come with questions that you don't 
know the answer to. That's okay! You can learn together. There is nothing wrong with not
knowing the answer to a question. This is about guiding the student to find the answer, even
if it is not immediately apparent to you. 

\end{itemize}
 





\end{document}


