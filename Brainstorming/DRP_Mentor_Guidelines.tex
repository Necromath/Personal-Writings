\documentclass{article}
\begin{document}
\author{Mauricio Montes}
\title{DRP Mentor Guidelines}

\maketitle

\section{What does it mean to be a mentor?}

According to the Cambridge Dictionary, the definition of mentor is "a person who gives a younger or
less experienced person help and advice over a period of time, especially at work or school". Your
job as a mentor is to guide your student through the process of self-learning. Notice that this
bears no expectation of you teaching the student. You are not expected to know everything. These
students signed up to this program to be guided by you, not to be taught by you. 


\section{Mentor Expectations}

Mentors are expected to meet with their students once a week for an hour and make sure that their
students are making progress on their projects. Mentors should let the organizers know ASAP about
any issue that arises such as a student missing meetings without notice or students not doing work.
Mentors are also expected to attend the presentations at the end of the semester. 

\subsection{Being an effective mentor}

\begin{itemize}


\item Be patient and understanding. Remember that the student is likely not as experienced as
you are.

\item Be encouraging. It is easy to get discouraged when learning new material. 

\item Be flexible. The student may not be interested in the project you initially
proposed. Be open to changing the project to better suit the student's
interests.

\item Be a good communicator. Make sure that you are communicating
your expectations clearly to the student.

\item Listening is more important than talking. Students might come with questions that you don't 
know the answer to. That's okay! You can learn together. There is nothing wrong with not
knowing the answer to a question. This is about guiding the student to find the answer, even
if it is not immediately apparent to you. 

\item Leverage the strengths of your mentee and their experience. If they seem confident with a
  topic that you are not familiar with, let them take the lead in your discussions. You are like the
  whetstone that sharpens the knife. You are not the knife. This does not mean you should give up on
  learning the topic with them though. Keep working on it together.

\end{itemize}

\section{Student Responsibilities}

As a mentor, you want to keep students on track. When your student suggests a topic, do a literature
review. Math stack exchange is a great place to start for literature requests. Make sure that the topic is not too easy or too hard for either of you. You want to be a
productive team. You also want to have a specific direction with the work you will be doing together
(i.e. a project, a proof of XYZ theorem, etc.). Try to find shiny gems that lie ahead in the
literature. This will keep you and your student motivated. Students don't need to be assigned
exercises or homework. But they should be expected to do some work on their own. They can show you
problems that they did, present a proof to you, or even just explain what their thoughts on
definitions, theorems, etc. are. Engagement is the most important aspect to learning. The more you
and the student engage with the material, the more you will learn. \\ Obviously, students have a
life outside of learning DRP. So the benchmark for success is not progress, but effort. If the
student is working hard, then great! If they are not, then you should let the organizer know.
Students are expected to put in at least 3 hours of work per week. This is a very arbitrary number,
but the point is that students should be putting in some effort. It should be enough hours to 
have a meaningful discussion with you during meetings. 

\section{What if I don't know the answer to a question?}

Questions don't require an immediate answer. 
This is a great opportunity to learn together. You are a mentor, you are supposed to use all your
resources to answer questions. Other textbooks, stack exchange, discussing with other students,
discussing with your student, etc. are all great resources. If you are still stuck, then you can
even ask a professor to help explain the topic to you. This is a great way to get to know your own
gaps in the knowledge with the material. Everyone should be learning during this process. Yourself
included! 

 
\end{document}


