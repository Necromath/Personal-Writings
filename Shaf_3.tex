\documentclass{article}
\usepackage{amsmath, amssymb, amsthm}

\title{Algebraic Geometry HW 4}
\author{Mauricio Montes}

\begin{document}

\maketitle

\section*{Exercise 1 Section 3 }

Solution: Given the closed set $X \subset \mathbb{A}^3$ defined by the equations
$f: x^2 - y^2 - z^2 + 1= 0$ and $g: x^2 + y^2 + z^2 = 0$. We wish to decompose $X$ into irreducible components.

The equations decompose in the following way: Note that moving $z$ to the other side in $f$ allows us to rewrite $g$ as
$2x^2 + 1 = 0$. Meaning that $x = \pm \frac{i}{\sqrt{2}}$. Replacing this relation in $g$, we get that $y^2 + z^2 = \frac{1}{2}$.

This means that $X$ is the union of the sets $X_1 = \{x = \pm \frac{i}{\sqrt{2}} \}, X_2 = \{y^2 + z^2 = \frac{1}{2}\}$

(I don't think so really. Union is supposed to be a product of these polynomials, so I need another formulation)

\section*{Exercise 4 Section 3}


Solution: We wish to decompose $X$ given by $y^2 = xz, z^2 = y^3$ into irreducible components. Substituting in 
the first equation in the second, we get that $z^2 = (xz)y$. Meaning $z = xy$. Plugging this back into our first
equation. We see that $y^2 = xz = x(xy) = x^2y$. Meaning $y = 0$ or $y = x^2$. Which then implies that $x^6 = z^2$ and 
$z = x^3$ or $z = -x^3$. This means that $X$ is the union of the sets 
$X_1 = \{y = 0\}, X_2 = \{y = x^2, z = x^3\}, X_3 = \{y = x^2, z = -x^3\}$. These irreducible components all come from clear
birational maps of $\mathbb{A}^1$ and so are all isomorphic to $\mathbb{A}^1$.

\section*{Exercise 3 Section 4}

Solution: As per the hint, studying the inclusion of $X$ in the affine variety $Y = \mathbb{A}^2$.
We have an induced isomorphism on the coordinate rings $\phi: k[Y]= k[x,y] \rightarrow k[X]$. To see why this is an isomorphism,
consider $f = \frac{g}{h} \in k[X]$. The condition of $f$ being regular is that $h$ does not vanish on $X$. However, if $h$ is not
a unit in $k[X]$, then there are infinitely many solutions to $h = 0$ in $X$. So then $h$ must be a unit if $f \in k[X]$.
But then this means that $f \in k[x,y] = k[Y]$. So thus $\phi$ is an isomorphism. However! The inclusion map is not an isomorphism.
Thus, $\mathbb{A}^2 - \mathbf{0}$ is not an affine variety.

\section*{Exercise 4 Section 4}

Solution: 


\section{Exercise 6 Section 4}

Solution: Supposing there was such a map $\phi$, then we would have an induced homomorphism 
$\phi^*: k[\mathbb{A}^n] \rightarrow k[\mathbb{P}^1]$.
However, the coordinate ring of the left hand side is $k[x_1, \ldots, x_n]$. But the right hand side is just $k$. This homomorphism
is a surjection, so it is achieved by sending sending variables to constants, i.e. quotienting out by $(x_i - c_i)$ for some
$c_i \in k$. Ideal variety correspondence then tells us that the map $\phi$ is then given by $C = (c_1, \ldots, c_n)$.


\end{document}

