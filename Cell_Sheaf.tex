\documentclass[12pt]{article}
\author{Mauricio Montes}
\usepackage{amsmath,amssymb,amsthm}

\newcommand{\F}{\mathcal{F}}
\newcommand{\<}{\langle}
\renewcommand{\>}{\rangle}

\title{Chemical Reaction Sheaf}

\begin{document}

\maketitle

\section{Introduction}

The purpose of this document is to track and monitor the development of the concept of
a sheaf of a chemical reaction network. This follows from the work of Hirono et al. in 
\cite{Hirono2021}. The authors mention of the interpretation of a chemical reaction network
as a hypergraph. This offers a new perspective of viewing chemical reaction networks, as 
\cite{duta2023sheaf} shows that hypergraphs admit a sheaf structure. The goal of this document
is to develop the concept of a sheaf of a chemical reaction network. Then, show that the 
sheaf associated to a chemical reaction network can be used to study the dynamics of the
chemical reaction network.

\section{Sheaf of a Chemical Reaction Network}

A chemical reaction network is defined as a quadruple, $(V, E, s, t)$, where $V$ is a set of
species (vertices), $E$ is a set of reactions (edges), $s: E \to V$ is the source map, and
$t: E \to V$ is the target map. The sheaf of a chemical reaction network is defined as follows.

The cellular sheaf $\F$ of a chemical reaction network is defined as a triple $\<\F(v), \F(e),
\F_{v \trianglelefteq e} \>$, where

\begin{itemize}
    \item $\F(v)$ is the stalk at $v \in V$,
    \item $\F(e)$ is the stalk at $e \in E$,
    \item $\F_{v \trianglelefteq e}$ is the restriction map.

\end{itemize}

We define the stalk for all $v \in V$, to be $\F(v) = \mathbb{R}^{|V|}$ and the stalk at $e \in E$ 
to be $\F(e_i) = \mathbb{R} / \text{Span}\{s(e_i), t(e_i)\}$, where $e_i$ is the $i$-th edge in $E$.
The restriction maps between sections of vertices and edges are exactly the quotient map.
This map is also relatively nice, $s(e_i)$ and $t(e_i)$ are linearly independent (pretty sure),
so the quotient map will always yield us a vector space.


Nobody seems to talk about the global sections of these sheaves for hypergraphs? 
From the definition, the global sections of the hypergraph will need to be defined.
It seems like the global sections of the hypergraph will be a choice of $x_v \in \F(v)$ for
all $v \in V$, such that $\F_{v \trianglelefteq e}(x_v) = \F_{u \trianglelefteq e}(x_u)$ for all
$e \in E$ such that $e = v - u$.

Need to make use of the hypergraph laplacian. In \cite{jost2018hypergraph}, the authors define
a hypergraph laplacian for the vertices and the hyperedges of a hypergraph.

In \cite{Hirono2021}, the authors define a morphism between CRN's and also define a reduction
morphism. This all seems to suggest that there would be an isomorphic relationship between the
sheaves defined over the CRN's. This is something that needs to be explored.

Also interesting: the author of \cite{MorphismsReaction} define a reactant morphism and a 
stoichiomorphism. This is all to say that if there exists a reactant morphism and a stochiomorphism
from one chemical reaction network to another, then there is an \textit{emulation} between the two
chemical reaction networks. This actually relates the differential systems of two chemical reaction
networks. 

So, my personal hypothesis is then that Emulation is a morphism of sheaves.

\bibliographystyle{plain}
\bibliography{Cell_Sheaf_Bib}
\end{document}


