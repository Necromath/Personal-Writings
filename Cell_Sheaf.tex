\documentclass[12pt]{article}
\author{Mauricio Montes}
\usepackage{amsmath,amssymb,amsthm, verbatim, tikz-cd, graphicx, quiver}
\theoremstyle{definition}
\newtheorem{definition}{Definition}
\newcommand{\F}{\mathcal{F}}
\newcommand{\<}{\langle}
\renewcommand{\>}{\rangle}

\title{Chemical Reaction Sheaf}

\begin{document}

\maketitle

\section{Introduction}

The purpose of this document is to track and monitor the development of the concept of
a sheaf of a chemical reaction network. This follows from the work of Hirono et al. in 
\cite{Hirono2021}. The authors mention of the interpretation of a chemical reaction network
as a hypergraph. This offers a new perspective of viewing chemical reaction networks, as 
\cite{duta2023sheaf} shows that hypergraphs admit a sheaf structure. The goal of this document
is to develop the concept of a sheaf of a chemical reaction network. Then, show that the 
sheaf associated to a chemical reaction network can be used to study the dynamics of the
chemical reaction network.

\begin{comment}% {{{

\section{Definitions}

\subsection{Topology}

\subsection{Simplicial Complexes}

\subsection{Simplicial Homology}

\subsection{Simplicial Cohomology}

\subsection{Chemical Reaction Network}

\end{comment}% }}}

\section{Goals for this paper and questions to be answered}
\begin{itemize}
  \item Define a chemical reaction sheaf
  \item How can this improve a problem?
  \item What needs to be done?
  \item Why is this useful?
  \item What are the implications of this?
\end{itemize}

\section{Chemical Reaction Network}
A chemical reaction network is defined as a quadruple, $\Gamma = (V, E, s, t)$, where $V$ is a set of
species (vertices), $E$ is a set of reactions (edges), $s: E \to V$ is the source map, and
$t: E \to V$ is the target map.
\section{Sheaf of a Chemical Reaction Network}
The sheaf of a chemical reaction network is defined as follows.

\begin{definition} Given a chemical reaction network $\Gamma$, the \textit{chemical reaction sheaf} associated to a 
network is the cellular sheaf $ \< \F(v) , \F(e), \F_{v \trianglelefteq e} \>$ where
(for all $v \in V $ and $e \in E$):

\begin{itemize}
  \item $\F(v) = \mathbb{R}[x_1, \ldots, x_{|V|}]$
  \item $\F(e) = \mathbb{R}[x_1, \ldots, x_{|V|}] / (t(e) - s(e))$
  \item $\F_{v \trianglelefteq e}$ is the quotient map by the ideal generated
    by $t(e) - s(e)$
\end{itemize}
\end{definition}

\section{Sheaf Cohomology of CRN}

We develop a theory of sheaf cohomology for these chemical reaction networks. 
\begin{definition}
  The \textit{kth sheaf cochain group}, is defined to be
  \begin{equation*}
    C^k(X ; \F) = \bigoplus_{dim(\sigma) = k} \F(\sigma)
  \end{equation*}
\end{definition}

Consider the following directed graph:
\[\begin{tikzcd}
	&& \bullet v_4 \\
	{} & \bullet v_1 & \bullet v_2 & \bullet v_3 & {}
	\arrow[from=1-3, to=2-2]
	\arrow[from=2-1, to=2-2]
	\arrow[from=2-2, to=2-3]
	\arrow[from=2-3, to=2-4]
	\arrow[from=2-4, to=1-3]
	\arrow[from=2-4, to=2-5]
\end{tikzcd}\]
This has a corresponding sheaf:

\newpage
\[\begin{tikzcd}
	&& \textcolor{rgb,255:red,92;green,92;blue,214}{\mathbb{R}[x_1,x_2,x_3,x_4] / (-x_3)} \\
	&& \textcolor{rgb,255:red,214;green,92;blue,92}{\mathbb{R}[x_1,x_2,x_3,x_4]} \\
	& \textcolor{rgb,255:red,92;green,92;blue,214}{\mathbb{R}[x_1,x_2,x_3,x_4] / (x_4 -x_3)} & \textcolor{rgb,255:red,92;green,92;blue,214}{\mathbb{R}[x_1,x_2,x_3,x_4]/(x_3-x_2)} \\
	\textcolor{rgb,255:red,214;green,92;blue,92}{\mathbb{R}[x_1,x_2,x_3,x_4]} && \textcolor{rgb,255:red,214;green,92;blue,92}{\mathbb{R}[x_1,x_2,x_3,x_4]} \\
	& \textcolor{rgb,255:red,92;green,92;blue,214}{\mathbb{R}[x_1,x_2,x_3,x_4] /(x_1 - x_4)} & \textcolor{rgb,255:red,92;green,92;blue,214}{\mathbb{R}[x_1,x_2,x_3,x_4] / (x_2 - x_1)} \\
	&& \textcolor{rgb,255:red,214;green,92;blue,92}{\mathbb{R}[x_1,x_2,x_3,x_4]} \\
	&& \textcolor{rgb,255:red,92;green,92;blue,214}{\mathbb{R}[x_1,x_2,x_3,x_4] / (x_1)}
	\arrow[from=2-3, to=1-3]
	\arrow[from=2-3, to=3-2]
	\arrow[from=2-3, to=3-3]
	\arrow[from=4-1, to=3-2]
	\arrow[from=4-1, to=5-2]
	\arrow[from=4-3, to=3-3]
	\arrow[from=4-3, to=5-3]
	\arrow[from=6-3, to=5-2]
	\arrow[from=6-3, to=5-3]
	\arrow[from=6-3, to=7-3]
\end{tikzcd}\]

One possible reduction of this graph would be collapsing the cells on the left down to one vertex.

The corresponding reduction will have a corresponding sheaf for the new CRN. Shown below:

\[\begin{tikzcd}
	& {\bullet v_4'} \\
	{} & {\bullet v_3'} & {}
	\arrow["{e_5}"', shift right=2, curve={height=6pt}, from=1-2, to=2-2]
	\arrow["{e_1}"', from=2-1, to=2-2]
	\arrow["{e_4}"', shift right=2, curve={height=6pt}, from=2-2, to=1-2]
	\arrow["{e_6}"', from=2-2, to=2-3]
\end{tikzcd}\]

This will have (a priori) the corresponding CRS:

\[\begin{tikzcd}
	& {\mathbb{R}[x_3',x_4']} \\
	{\mathbb{R}[x_3',x_4'] / (x_3' - x_4')} && {\mathbb{R}[x_3',x_4'] / (x_4' - x_3')} \\
	{\mathbb{R}[x_3',x_4'] /(+x_3')} & {\mathbb{R}[x_3',x_4']} & {\mathbb{R}[x_3',x_4'] / (-x_3')}
	\arrow[from=1-2, to=2-1]
	\arrow[from=1-2, to=2-3]
	\arrow[from=3-2, to=2-1]
	\arrow[from=3-2, to=2-3]
	\arrow[from=3-2, to=3-1]
	\arrow[from=3-2, to=3-3]
\end{tikzcd}\]

However, when we consider the direct image sheaf given by the map 
$\varphi: X \to X'$, $\varphi(v_1) = v_3', \varphi(v_2) = v_3', \varphi(v_3) = v_3', \varphi(e_2) = e_1 , \varphi(e_3) = e_1$ and
the identity elsewhere. We get: a small edit over yonder 






\begin{comment}% {{{
\section{My Thoughts}
Nobody seems to talk about the global sections of these sheaves for hypergraphs? 
From the definition, the global sections of the hypergraph will need to be defined.
It seems like the global sections of the hypergraph will be a choice of $x_v \in \F(v)$ for
all $v \in V$, such that $\F_{v \trianglelefteq e}(x_v) = \F_{u \trianglelefteq e}(x_u)$ for all
$e \in E$ such that $e = v - u$.

In \cite{jakobhansen}, the author mentions that if a graph $G$ is connected with at least one cycle
and all the stalks have the same dimension, then almost any choice of restrictions will yield a sheaf 
with no global sections. This is bad news for us if we want to use sheaf cohomology for these chemical
reaction networks. This is because sheaf cohomology is concerned with the global sections of a sheaf.

However, in \cite{flowsheaf} the authors do mention that that sheaves can encode linear transformations
at nodes, which would accommodate reaction balancing! This all seems great and promising.

Need to make use of the hypergraph laplacian. In \cite{jost2018hypergraph}, the authors define
a hypergraph laplacian for the vertices and the hyperedges of a hypergraph.

In \cite{Hirono2021}, the authors define a morphism between CRN's and also define a reduction
morphism. This all seems to suggest that there would be an isomorphic relationship between the
sheaves defined over the CRN's. This is something that needs to be explored.

Also interesting: the author of \cite{MorphismsReaction} define a reactant morphism and a 
stoichiomorphism. This is all to say that if there exists a reactant morphism and a stochiomorphism
from one chemical reaction network to another, then there is an \textit{emulation} between the two
chemical reaction networks. This actually relates the differential systems of two chemical reaction
networks. This is a bit too much at this point.
\end{comment}
% }}}


\bibliographystyle{plain}
\bibliography{Cell_Sheaf_Bib}
\end{document}


