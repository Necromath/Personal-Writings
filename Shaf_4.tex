\documentclass[12pt]{article}

\usepackage{amsmath,amssymb,amsthm,amsfonts}

\title{Algebraic Geometry HW 5}

\author{Mauricio Montes}

\begin{document}

\maketitle

\section{Exercise 4.1}

Solution: $(\Rightarrow)$  Suppose we have an affine, irreducible variety that is of the form $U = V(I(f_1, \ldots, f_m))$. 
Homogenizing these polynomials wrt $x_i$ for each $i$ we get 
\begin{equation*}
\{ F_1, \ldots, \hat{F_i}, \ldots, F_m \}
\end{equation*}
where $\hat{F_i}$ means that $F_i$ is omitted.  We can then write $U_i = V(I(F_1, \ldots, F_m))$ which is a closed projective set corresponding
to $U$. Their (finite) union forms the projective variety $V(I(F_1, \ldots, F_m))$ which is the projective closure of $U$.
If this was not irreducible, then there would be 2 nonempty closed sets that are disjoint and whose union is $\bar{U}$. These
would in turn correspond to 2 nonempty closed sets in $U$ that are disjoint and whose union is $U$ by the ideal variety correspondence.
This is a contradiction of the irreducibility of $U$

$(\Leftarrow)$ Suppose we have an irreducible projective closure $\bar{U}$ of a variety. Being irreducible means that the ideal corresponding to
the variety is prime. The ideal variety correspondence then applies when we restrict to the affine part of the projective variety. This means that on 
each of the affine charts, the variety is irreducible. As per equation (1.29) from the book, we see that $U$ is recovered by intersecting the projective
closure with the affine charts. This means that $U$ is irreducible.

\section{Exercise 4.2}

Solution: This association is the one given by $U \mapsto \bar{U}$ and $\bar{U} \mapsto U \cap \mathbb{A}^n _0$. When $U$ is affine, this corresponds to the
definition associated in the book. This follows almost exactly from the book? See page 46. "The homogenous equations of $\bar{U}$ are of the form
\begin{equation*}
  S^k _0 F(S_1 / S_0, \ldots, S_n / S_0)
\end{equation*}"
The injection from $U$ to $\bar{U}$ yields a correspondence for any affine subvariety to the ideal of the associated projective subvariety. Ideal-variety correspondence
then gives us the desired result. 

\section{Exercise 4.5}

Solution: 


\end{document}
