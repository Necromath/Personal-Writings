% This document is going to serve as a list of definitions for the work that I am doing in
% topological signal processing. This is going to be a living document that I will update it as I
% learn more useful things.

\documentclass[12pt]{article}
\author{Mauricio Montes}

\usepackage{amsmath, graphicx, amssymb, amsthm}

\theoremstyle{plain}
\newtheorem{thm}{Theorem}[section]

\theoremstyle{definition}
\newtheorem{definition}[thm]{Definition}
\newtheorem{exmp}[thm]{Example}

\title{TDA Introduction and some other concepts}

\begin{document}

\maketitle

\section{Introduction}

This document is for me to keep track of the definitions and ideas I have about topological signal
processing. This will probably get long, so maybe we can chunk it later. I'm not sure if I want to
start all the way with the definition of a topological space. But I think I can start with the ideas
of homology and cohomology. 


\section{Definitions}

\subsection{Topology Preliminaries}

\begin{definition}[Topology] Given a set $X$, a \textit{topology} on $X$ is a set $\mathcal{T} \subset \mathcal{P}(X)$ of subsets
of $X$. The elements of a topology are called \textit{open sets}. To be a topology, certain
requirements must be satisfied. Namely :

\begin{itemize}

  \item $\phi, X \in \mathcal{T}$

  \item For any collection of open sets $\{U_\alpha \}_{\alpha \in I}$, $\bigcup_{\alpha \in I} U_\alpha$

  \item For any finite collection of open sets $\{U_\alpha\}_{\alpha \leq n}$,$\bigcap_{\alpha} U_\alpha$
    
\end{itemize}

\end{definition}

\subsection{Simplicial Complexes}

The central objects of simplicial homology are simplicial complexes. The purpose of all of this is so that we can associate a
topological space with a chain complex, and then we can do algebra with the chain complex to extract topological information about the space.


\begin{definition}[Simplices and Simplicial Complexes]

A k\textit{-simplex} is the convex hull of a set of $k+1$ points in some Euclidean space. We can think of a 0-simplex as a
  vertex. A 1-simplex is an edge, a 2-simplex is a triangle, and so on. 

  Note that \begin{itemize}
    \item a k-simplex has $k+1$ faces, which are the simplices of dimension $k-1$ that are
  contained in it.

    \item A \textit{simplicial complex} is a collection of simplices such that the intersection of any two simplices is either empty or
another simplex.

    \item The \textit{dimension} of a simplicial complex is the maximum dimension of any of the simplices in the complex. 

      \end{itemize}

\end{definition}

\subsection{Simplicial Homology}

In order to do algebra with simplicial complexes, we need to associate it to algebraic objects that
we can manipulate. This is where the chain complex comes in.

\begin{definition}[Chain Group]  Say $X$ is a $k$-dimensional simplicial complex. The \textit{chain group} $C_k(X,\mathbb{R})$
is the vector space (over $\mathbb{R}$) with basis elements given by the number of $k$-simplices in $X$.
\end{definition}

Below is an example that we will use to illustrate the concept of a chain group.

\begin{figure}[ht]
  \begin{center}
      \includegraphics[width=0.5\textwidth]{Simp_Cx.jpg}
  \end{center}
\caption{A simplicial complex, with 6 vertices and 7 edges.}
\end{figure}

Inspecting the image, we see that we have 6 vertices and 7 edges and no higher dimensional simplices. 
We can write down the chain groups associated with this complex: $C_1(X, \mathbb{R}) = \mathbb{R}^6 , C_2(X, \mathbb{R}) = \mathbb{R}^7$


Note that $C_k(X, \mathbb{R}) = 0, k > 2$. For our purposes, we will use $\mathbb{R}$ for our base field for the 
associated vector space and will suppress writing down the associated field for the chain group. Writing 
$C_k(X)$ instead for brevity.

\begin{definition}[Chain Maps and Chain Complexes]Between chain groups, there exists a map $\delta_{k}: C_k(X) \to C_{k-1}(X)$ given by
\begin{equation}
  \delta_k (v_{i_1}, \ldots v_{i_k}) =  \sum_{j = 0}^{k} (-1)^j (v_{i_1}, \ldots, \hat{v_{i_j}} , \ldots, v_{i_k})
\end{equation}

This is called the \textit{kth chain map}. A set pair of chain groups and chain maps form a \textit{chain complex}.
The associated chain maps have the property that $\delta_{k-1} \circ \delta_{k} = 0$. 

\end{definition}

As with any map between vector spaces, it is always an object of interest to find the kernel and image of the map. What we do instead
is a bit different. We compare the kernel of the map to the image of the previous map. This is the idea behind the homology groups.

\begin{definition}[Homology Groups]

Given a chain complex $C(X)$, we define the \textit{kth homology group} to be 

\begin{equation}
H_k(C) = ker(\delta_k) / Im(\delta_{k+1})
\end{equation}

\end{definition}

\subsubsection{A helpful example}

Below, we have the result of the boundary map acting on the chain group defined by 
our edges. With blue indicating a positive value of the basis element of our chain 
group and red indicating a negative. We are interested in sums of chains whose boundary
maps equal to zero. These would correspond to "holes" in the picture of our simplicial complex.


\begin{figure}[ht]
  \begin{center}
      \includegraphics[scale = .4]{Colored_Cx.jpg}
  \end{center}
\caption{The boundary map of the edges of the simplicial complex}
\end{figure}

Let's perform a calculation, following our formula (and our picture), we see that

\begin{align*}
\partial(e_1) = v_2 - v_1 \\
\partial(e_2) = v_3 - v_2 \\
\partial(e_3) = v_3 - v_1
\end{align*}

Since the map $\partial$ is linear (it is a map between vector spaces), we can write
down 
\begin{equation*}
\partial(e_1 + e_2 - e_3) = \partial(e_1) + \partial(e_2) - \partial(e_3) = 0
\end{equation*}

This means that the element $(e_1 + e_2 - e_3) \in C_1(X)$ is actually inside $ker(\partial_1)$ !
Meaning that in the definition of homology, this is a nontrivial element! From our picture, we also
see that $(e_5 + e_7 - e_6) \in ker(\partial_1)$. This gives 2 nontrivial elements in the kernel of $\partial_1$.

It is important to note that in the image, there is no higher dimensional 2-simplex that these 1-simplices are the
boundary of, meaning that $Im(\partial_2) = 0$. Writing this all down, we mean to say:

\begin{equation*}
  H_1(X) = \mathbb{R}^2
\end{equation*}

We can augment this example by adding a 2-simplex to our simplicial complex. One that would fill in one of the holes
in the complex. Then the homology group would be $H_1(X) = \mathbb{R}^1$. If we attached a 3-simplex to the complex, 
then $H_1(X) = \mathbb{R}^1$. But then $H_2(X) = \mathbb{R}^1$ as well! This is due to the creation of a "void" in the
tetrahedron that we attached.  This is the essence of simplicial homology.

Finally, some interpretations / intuition of homology groups:

\begin{itemize}

\item $H_k(X) = 0$ for $k > dim(X)$

\item $H_0(X)$ is the number of connected components of $X$

\item $H_1(X)$ is the number of "holes" in $X$

\item $H_2(X)$ is the number of "voids" in $X$

\item If $X, Y$ are simplicial complexes, and $f: X \to Y$ is a simplicial map, then $f$ induces a map $f_*: H_k(X) \to H_k(Y)$

\end{itemize}


\subsection{Simplicial Cohomology}

Fill me

\subsection{Persistent Homology}

Fill me

\subsection{Hodge Laplacian}

Fill me 

\subsection{Sheaves}

Fill me 

\subsection{Sheaf graph networks}

Fill me 

\subsection{Sheaf Cohomology}

Fill me

\subsection{Further Directions and Questions}

\begin{itemize}
  \item Is there a theory about probability sheaves? Could one be written down such that
    the maps between sheaves would somehow communicate information about the probability 
    states of network? (Does this question even make sense)

  \item What direct applications could this have towards signal processing? Is there an immediate
    impact besides the answer from Robinson "In theory we can"

  \item What is some nice data that TSP can be used on?

  \item The current methods from TDA and TSP are: Persistent Homology, Bottleneck/Wasserstein distance, 
    0 dimensional sublevel set persistence, 

  \item Homology for lattice configurations?

  \item Homology of dynamical systems

  \item Given a coboundary map $\delta$, we can construct a laplacian matrix associated with a sheaf.

  \item Chemical Reaction networks have a hypergraph structure. Is there a way to write down a sheaf and associated
    laplacian matrix with this hypergraph?

\end{itemize}

\end{document}
