%% LANL Document here

\documentclass{beamer}

\begin{document}

\begin{frame}
  \frametitle{Topological Data Analysis (TDA) Pipeline}

  \begin{itemize}
    \setlength\itemsep{1em}
    \item \textbf{Data Representation:} Represent the dataset as a point cloud in a high-dimensional space with a distance metric.
    \item \textbf{Simplicial Complex Construction:}
      \begin{itemize}
        \setlength\itemsep{0.5em}
        \item Connect points to form simplices.
        \item Aggregate simplices into a complex.
      \end{itemize}
    \item \textbf{Filtration:} Introduce a filtration parameter "t" and vary it.
    \item \textbf{Persistence:}
      \begin{itemize}
        \setlength\itemsep{0.5em}
        \item Track topological changes during filtration.
        \item Capture birth and death of topological features.
      \end{itemize}
  \end{itemize}
\end{frame}

\begin{frame}
  \frametitle{Topological Data Analysis (TDA) Pipeline (cont.)}

  \begin{itemize}
    \setlength\itemsep{1em}
    \item \textbf{Topological Summarization:} Summarize features using barcodes or persistence diagrams.
    \item \textbf{Interpretation and Visualization:}
      \begin{itemize}
        \setlength\itemsep{0.5em}
        \item Interpret results in the context of the original data.
        \item Visualize persistent homology using landscapes, heatmaps, etc.
      \end{itemize}
    \item \textbf{Validation and Application:}
      \begin{itemize}
        \setlength\itemsep{0.5em}
        \item Validate results against domain knowledge.
        \item Apply findings for insights into dataset structure.
      \end{itemize}
    \item \textbf{Integration:}
      \begin{itemize}
        \setlength\itemsep{0.5em}
        \item Simplicial complex is a crucial part of the analysis.
        \item Iteratively refine complex and persistent homology for multi-scale exploration.
      \end{itemize}
  \end{itemize}
\end{frame}

\begin{frame}
  \frametitle{TDA for Wave Data Pipeline}

  \begin{itemize}
    \setlength\itemsep{1em} % Adjust the separation between items
    \item \textbf{Data Representation:} Represent wave data as point clouds in a high-dimensional space.
    \item \textbf{Simplicial Complex Construction:}
      \begin{itemize}
        \setlength\itemsep{0.5em} % Adjust the separation between sub-items
        \item Connect points using subsampling methods.
        \item Aggregate simplices into a simplicial complex.
      \end{itemize}
    \item \textbf{Lower Star Filtration:} Apply a lower star filtration method.
    \item \textbf{Persistence:}
      \begin{itemize}
        \setlength\itemsep{0.5em} % Adjust the separation between sub-items
        \item Track changes during the filtration.
        \item Capture birth and death of topological features using persistent homology.
      \end{itemize}
  \end{itemize}
\end{frame}

\begin{frame}
  \frametitle{TDA for Wave Data Pipeline (cont.)}

  \begin{itemize}
    \setlength\itemsep{1em} % Adjust the separation between items
    \item \textbf{Topological Summarization:} Summarize persistent features using persistence landscapes.
    \item \textbf{Interpretation and Visualization:}
      \begin{itemize}
        \setlength\itemsep{0.5em} % Adjust the separation between sub-items
        \item Interpret results in the context of wave data.
        \item Visualize persistent homology using barcodes, persistence diagrams, etc.
      \end{itemize}
    \item \textbf{Validation and Application:}
      \begin{itemize}
        \setlength\itemsep{0.5em} % Adjust the separation between sub-items
        \item "Persistent" topological features should correspond to wave features of high amplitude.
        \item The physical reality (obstructions) of the data should be reflected in the topological features of our data.
      \end{itemize}
    \item \textbf{Integration:}
      \begin{itemize}
        \setlength\itemsep{0.5em} % Adjust the separation between sub-items
        \item Subsampling used for simplicial complex construction.
        \item Incorporate in ML pipeline for obstruction / feature detection in noisy data.
      \end{itemize}
  \end{itemize}
\end{frame}



\end{document}

