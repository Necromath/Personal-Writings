\documentclass{beamer}
%Information to be included in the title page:
\title{What is the shape of data?}
\subtitle{A brief introduction to topological data analysis}
\author{Mauricio Montes}
\institute{Auburn University}
\date{\today}

\begin{document}

\frame{\titlepage}

% Introduction
\begin{frame}% {{{

\frametitle{What is TDA?}

TDA stands for Topological Data Analysis.

\pause

\begin{itemize}

\item Topology is the study of geometric properties preserved under continuous deformations.

\pause

\item This means that we can stretch, bend, and deform objects, but we cannot tear or glue them.

\pause

\item Data analysis is the process of studying and modeling data to extract useful information.

\pause

\item Typically, we use statistics and machine learning to analyze data. We can
improve these methods by using tools from topology. 

\end{itemize}

% Show a picture of a donut and coffee mug transformation and a picture of a SVM.


\end{frame}
% }}}

% Intuition
\begin{frame}% {{{

\frametitle{Developing our intuition}

\pause

%Oct1,2023: Need topological definitons, open set. Simpicial complex, recall the
%paper from chazal

\begin{itemize}

\item One of the core ideas of TDA is that our data is sampled from a manifold.

\pause

\item We can think of manifolds as "things that locally look like"
$\mathbb{R}^n$. Most objects in the real world are manifolds. 

\pause

\item If we take enough samples, we can approximate the shape of the manifold our data is living in.
  This allows us to answer topological questions about our data. 

\end{itemize}

\end{frame}
% }}}

% TDA Pipeline
\begin{frame}% {{{
\frametitle{The TDA Pipeline}

\begin{itemize}

\item The input is a finite set of points with some notion of similarity between them. This could be
  distance, correlation, or some other metric.
%want examples of correlation metrics between points
\pause

\item We construct a simplicial complex from the data. This is a combinatorial object that encodes
  the shape of the data.
%Make some comment about how you can think of a simplex as a triangle, tetrahedron, etc.
\pause

\item We compute topological invariants of the simplicial complex. This would be something like the
  Euler characteristic, homology, or persistent homology of the complex.

\pause

\item This new topological data provides us with a new descriptor of our original dataset. This can then be used in
  conjunction with other methods to improve our analysis.
%Give an example of how this can be used to improve analysis. Refer to an SVM or something.


\end{itemize}

\end{frame}% }}}

% Simplical Complexes
\begin{frame}% {{{

\frametitle{Simplicial Complexes}

\begin{itemize}

\item The problem with data is that it is discrete, but manifolds are continuous.

\pause

\item We can remedy this by building a simplicial complex from our data to approximate the manifold.

\pause

\item A $k$-simplex is the convex hull of $k+1$ linearly independent points in $\mathbb{R}^n$.
%Show a picture of a 0,1,2,3 simplex
\pause

\item A simplicial complex $K$ is a collection of simplices that satisfies two conditions:
\end{itemize}

\begin{enumerate}

\item If $\sigma \in K$, then every face of $\sigma$ is also in $K$.

\item If $\sigma_1, \sigma_2 \in K$, then $\sigma_1 \cap \sigma_2$ is either empty or a face of both
  $\sigma_1$ and $\sigma_2$.
\end{enumerate}

%Show a picture of a simplicial complex

\end{frame}


% }}}

% Simplicial Homology

\begin{frame}% {{{
  \frametitle{Simplicial Hole-mology}

  \begin{itemize}
    \item Say we want to know how many $n$-dimensional holes a simplicial complex has.
      \pause
    \item We can intuitively think of an $n$-dimensional hole as a gap that is captured by the
      boundary of an $n+1$ simplex, alternatively by a combination of $n$-simplices. 
     \pause
   \item Let $C_0, C_1, \ldots, C_k, \ldots$ be groups isomorphic to the integers, with $C_i$ having
     one copy of $\mathbb{Z}$ per $k$-simplex in our simplicial complex. 
     \pause
   \item Notice that we can recover a set of $k-1$-simplices from a $k$-simplex by looking at its
     boundary.

 \end{itemize}

\end{frame}% }}}

% Simplicial Homology pt 2

\begin{frame}
  \frametitle{Simplicial Holemology 2 : Electric Boogaloo}

\end{frame}

% Vietoris Rips Complex, Alpha Complex, Cech Complex

\begin{frame}
  \frametitle{Objects of Interest}
  \begin{itemize}
    \item hi 
  


  \end{itemize}


\end{frame}

% Topology, need to cut down
%Topological Preliminaries{{{

\begin{frame}% {{{

\frametitle{Topological Preliminaries}

\begin{itemize}

\item A topological space is a pair $(X,\tau)$ consisting of a set $X$ with a collection of subsets
  $\tau$, called the topology on $X$, such that:

\begin{enumerate}

\item $\emptyset, X \in \tau$

\item If $U_i \in \tau$ for $i \in I$, then $\bigcup_{i \in I} U_i \in \tau$

\item If $U_1, U_2 \in \tau$, then $U_1 \cap U_2 \in \tau$

\end{enumerate}

\item The Euclidean topology on $\mathbb{R}^n$ is the collection of all open subsets of
  $\mathbb{R}^n$. We define an open set as a set $B_r(x) = \{y \in \mathbb{R}^n : d(x,y) <r \} $.
  Where $d$ is the Euclidean metric. 


\end{itemize}


\end{frame}
% }}}
%def of a topology
\begin{frame}% {{{

\frametitle{Topological Preliminaries}

\begin{itemize}

\item We say that a topological space is Hausdorff if for any two points $x,y \in X$ there exist
  disjoint open sets $U,V \in \tau$ such that $x \in U$ and $y \in V$.

\pause

\item A basis for a topological space $(X,\tau)$ is a collection $\mathcal{B}$ of open
  sets such that every open set in $\tau$ can be written as a union of sets in $\mathcal{B}$. 

\item A topological space is second countable if it has a countable basis.

\pause

\item In the Euclidean topology, the topology is actually generated by the collection of open balls.
These are a basis for the Euclidean topology.

\end{itemize}

\end{frame}
% }}}
%def of hdoff, basis, second countable, and open balls
\begin{frame}% {{{

\frametitle{Topological Preliminaries}

\begin{itemize}

\item The fundamental map between topological spaces is the Homeomorphism.

\pause

\item A homeomorphism is a continuous map $f: X \to Y$ such that there exists a continuous map
  $g: Y \to X$ such that $g \circ f = id_X$ and $f \circ g = id_Y$. (Bicontinuous, Bijective)

\pause

\item If there exists a homeomorphism between two topological spaces, we say that they are
  homeomorphic.

\end{itemize}

\end{frame}
% }}}
%Homeomorphism def
\begin{frame}% {{{

\frametitle{Topological Preliminaries}

\begin{itemize}

\item We can finally define a manifold.

\pause

\item A topological manifold is a second countable Hausdorff space that is locally
  homeomorphic to $\mathbb{R}^n$.

\pause

\item The last condition means that for every point $x \in X$, there exists an open set $U \in \tau$
  such that $x \in U$ and $U$ is homeomorphic to an open set $\hat{U} \subseteq \mathbb{R}^n$.

\item To make this a smooth manifold, we require that the homeomorphism is actually a
  diffeomorphism. (Smooth map with smooth inverse)

\end{itemize}

\end{frame}
% }}}
%Definition of a manifold

% }}}




\end{document}
